\documentclass[10pt,a4paper,twoside]{article}
\usepackage[dutch]{babel}
\usepackage{graphicx}
\usepackage{float,flafter}
\usepackage{hyperref}
\usepackage{inputenc}

\setlength\paperwidth{20.999cm}\setlength\paperheight{29.699cm}\setlength\voffset{-1in}\setlength\hoffset{-1in}\setlength\topmargin{1.499cm}\setlength\headheight{12pt}\setlength\headsep{0cm}\setlength\footskip{1.131cm}\setlength\textheight{25cm}\setlength\oddsidemargin{2.499cm}\setlength\textwidth{15.999cm}

\begin{document}

\begin{center}
\hrule

\vspace{.4cm}
{\bf {\Large TERM PAPER  }}\\
\vspace{.3cm}
{\bf {\huge Big data management challenges in healthcare }}
\vspace{.3cm}
\end{center}

{\bf Name:}  Priyanshu Gupta

{\bf Roll no:}  19111042 

{\bf Branch: } 6th Biomedical Engineering, 2022 
\\
\hrule

\vspace{.3cm}
\section*{Introduction} 
What is big data? “Big data” is a term that was introduced in the 1990s to include data sets too large to be used with common software. It was defined as information assets characterized by high volume, velocity, and variety that required specific technology and analytic methods for its transformation into use. In addition to the three attributes of volume, velocity, and variety, some have suggested that for big data to be effective, nuances including quality, veracity, and value need to be added as well. 
Big data management is a critical challenge across health research disciplines. Data from clinical studies, research and observations of individuals’ lifestyle and environmental exposure are all of importance in advanced health research. This data reality has raised pressing demands for an enhanced data management capacity beyond what traditional approaches can deliver. Big data originated from disparate sources is inherently heterogeneous and not connected. 



\end{document}